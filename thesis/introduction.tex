\section{序論}

企業がマーケティングのために行う商品の評判分析において、
商品レビューに対するレーティング予測は重要な要素技術のひとつである。
何万件という大量のレビューデータを人手で処理することは難しく、
計算機による自動化が望まれる。
その中で商品を複数のカテゴリにおいてレーティング予測をする問題がある。
カテゴリとは、宿泊施設のレビューを例にすると、サービス、立地、食事等の
レーティングが付けられる各項目のことである。
この問題に関する従来手法\cite{fujitani15}は、文間の関係性を考慮しておらず、
カテゴリ間については考慮しているものの複雑な関係性を捉えられていない。

近年、その評判分類において、ニューラルネットワークを用いた手法
\cite{nal14,rie14,duyu15}が
提案されており、従来の手法を上回る正答率を達成している。
ニューラルネットワークをレーティング予測に用いる利点はまず
層の数を増やすことによって入力の深い繋がりを考慮できることである。
例えば、文毎の素性を入力とすれば文間の関係性を捉えることができる。
さらに、多カテゴリの分類問題においてはカテゴリ間の関係性を捉えた分類が
実現できる。
しかし、レーティング予測に関する多くの研究は1つのカテゴリにおける
二値分類問題を対象としている。

文や文書の意味表現の学習手法として、単語と文書の分散表現を同時に学習する
パラグラフベクトル\cite{quoc14}がある。
これは評判分類問題に対して優れた性能を示している。
しかし、文書全体にパラグラフベクトルを用いた場合、
レーティングの予測時に文の位置関係を考慮できない。

本研究は、複数カテゴリにおける評判分類について、
文書及び文間の関係とカテゴリ間の関係を同時に考慮した分類の実現を目的とする。

提案手法では、まず、パラグラフベクトル\cite{quoc14}によって
各レビューの文書ベクトルと文ベクトルをする。
次に、その文書ベクトルと重み付け平均された文ベクトルを用いて
ニューラルネットワークによる分類器において分類しレーティング予測を行う。
楽天トラベルのデータセットを用いた実験において、
提案手法は従来手法\cite{fujitani15}に対して約2pp上回る正答率を
示した。
レーティングの平均二乗誤差(RMSE)を元にした評価基準では、
従来手法\cite{fujitani15}において欠点となっていたカテゴリについて
それを上回る結果を示した。
