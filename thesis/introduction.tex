\section{序論} \label{sec:Introduction}

本研究は多カテゴリにおけるレーティング予測に関する研究である。
以下に、研究背景と目的、及び、提案手法の概略、本研究の貢献を示す。
最後に論文全体の構成を示す。

\subsection{背景}

企業がマーケティングのために行う商品の評判分析において、
商品レビューに対するレーティング予測は重要な要素技術のひとつである。
何万件という大量のレビューデータを人手で処理することは難しく、
計算機による自動化が望まれる。
その中で商品を複数のカテゴリにおいてレーティング予測をする問題がある。
カテゴリとは、宿泊施設のレビューを例にすると、サービス、立地、食事等の
レーティングが付けられる各項目のことである。
この問題に関する従来手法\cite{fujitani15}は、文間の関係性を考慮しておらず、
カテゴリ間については考慮しているものの複雑な関係を考慮できていない。

近年、そのレーティング予測において、ニューラルネットワークを用いた手法
\cite{nal14,rie14,duyu15}が
提案されており、従来の手法を上回る正答率を達成している。
ニューラルネットワークをレーティング予測に用いる利点はまず
層の数を増やすことによって入力の複雑な繋がりを考慮できることである。
例えば、文毎の素性を入力とすれば文間の関係を考慮することができる。
さらに、多カテゴリにおけるレーティング予測では
カテゴリ間の関係を考慮した予測が実現できる。
しかし、レーティング予測に関する多くの研究は1つのカテゴリにおける
二値または多値分類問題を対象としている。

文や文書の意味表現の学習手法として、単語と文書の分散表現を同時に学習する
パラグラフベクトル\cite{quoc14}がある。
これはレーティング予測において優れた性能を示している。
しかし、文書全体にパラグラフベクトルを用いた場合、
レーティングの予測時に文間の関係を考慮できない。


\subsection{目的}

複数カテゴリにおけるレーティング予測について、
文書及び文間の関係とカテゴリ間の関係を同時に考慮したものの実現を目的とする。
これにより、提案手法が従来手法\cite{fujitani15}より高い正答率を
達成することを目指す。
また、実際に文書・文間の関係の考慮がレーティング予測の正答率向上に
有効であるか検証する。


\subsection{提案手法}

提案手法はパラグラフベクトルと\nn を用いたレーティング予測の手法である。
提案手法では、まずパラグラフベクトル\cite{quoc14}によって
各レビューの文書ベクトルと文ベクトルを生成する。
次に、各レビューの文ベクトルをその位置によって重み付け平均する。
これにより、文の大まかな位置関係を保持したまま
レビュー間の文ベクトルの数を固定にする。
最後に、その文書ベクトルと重み付け平均された文ベクトルを
\nn による分類器によって分類しレーティング予測を行う。


\subsection{貢献}

本研究の貢献は2つある。
1つ目は、レーティングの付けられていない
一般の商品の批評文書から多カテゴリにおけるレーティングを、
従来手法より高い正答率で予測できるようになったことである。
これによって、企業は多様な文書から自社の商品のレーティングを複数の観点で
分析できる。
2つ目は、ユーザが付与したものより客観的なレーティングを
各商品レビューに対して予測できることである。
各商品レビューのレーティングはそれぞれのユーザが付与するため、
個々のユーザの主観による影響が大きい。
しかし、機械学習によるレーティング予測では、
大量の商品レビューから学習した結果から
各レビューに対してレーティングを予測する。
そのため、ユーザの主観による影響を大きく減らすことができる。
これは、企業が個々の商品レビューを分析する際に役立つ。

%本研究は、多カテゴリにおけるレーティング予測について、
%従来手法\cite{fujitani15}より高い正答率を示す手法を提案した。
%提案手法は、レーティング不可能という意味を持った
%レーティング値の予測において従来手法より特に優れていることが分かった。
%また、レーティング予測において文書・文ベクトルを同時に用いることが
%有効であることを示した。
%これは同時に、文書ベクトルと文ベクトルがいくらか異なる特徴を捉えていることを
%示す。
%さらに、文ベクトルの位置関係の考慮がレーティング予測に重要であることを示した。

%\subsection{構成}
%
%本論文は本節を含む6節からなる。
%%\secref{sec:Introduction}は本節であり、本研究の背景、目的、
%%研究分野への貢献、論文の全体の構成について述べる。
%\secref{sec:RelatedResearch}では、本研究の従来手法\cite{fujitani15}と、
%提案手法がレーティング予測のために用いるいくつかの手法、研究について述べる。
%\secref{sec:ProposedMethod}では、提案手法の基礎となる考えと
%その具体的なアルゴリズムについて説明する。
%\secref{sec:Experiments}では、提案手法と従来手法、及び、3つの比較手法を
%用いた実験の実験設定と結果について説明する。
%\secref{sec:Discussion}では、実験結果とそれによって明らかとなった
%提案手法の性質について考察する。
%また、提案手法の問題点からその改善方法について議論する。
%\secref{sec:Conclusion}では、まとめと今後の予定について述べる。
