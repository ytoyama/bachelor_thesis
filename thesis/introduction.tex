\section{序論} \label{sec:Introduction}

\subsection{背景}

企業がマーケティングのために行う商品の評判分析において、
商品レビューに対するレーティング予測は重要な要素技術のひとつである。
何万件という大量のレビューデータを人手で処理することは難しく、
計算機による自動化が望まれる。
その中で商品を複数のカテゴリにおいてレーティング予測をする問題がある。
カテゴリとは、宿泊施設のレビューを例にすると、サービス、立地、食事等の
レーティングが付けられる各項目のことである。
この問題に関する従来手法\cite{fujitani15}は、文間の関係性を考慮しておらず、
カテゴリ間については考慮しているものの複雑な関係性を捉えられていない。

近年、そのレーティング予測において、ニューラルネットワークを用いた手法
\cite{nal14,rie14,duyu15}が
提案されており、従来の手法を上回る正答率を達成している。
ニューラルネットワークをレーティング予測に用いる利点はまず
層の数を増やすことによって入力の深い繋がりを考慮できることである。
例えば、文毎の素性を入力とすれば文間の関係性を捉えることができる。
さらに、多カテゴリのレーティング予測においては
カテゴリ間の関係を考慮した予測が実現できる。
しかし、レーティング予測に関する多くの研究は1つのカテゴリにおける
二値分類問題を対象としている。

文や文書の意味表現の学習手法として、単語と文書の分散表現を同時に学習する
パラグラフベクトル\cite{quoc14}がある。
これはレーティング予測において優れた性能を示している。
しかし、文書全体にパラグラフベクトルを用いた場合、
レーティングの予測時に文の位置関係を考慮できない。


\subsection{目的}

本研究は、複数カテゴリにおけるレーティング予測について、
文書及び文間の関係とカテゴリ間の関係を同時に考慮した予測の実現を目的とする。
これにより、提案手法が従来手法\cite{fujitani15}より高い正答率を
達成することを目指す。
目的を達成するため、提案手法ではまずパラグラフベクトル\cite{quoc14}によって
各レビューの文書ベクトルと文ベクトルをする。
次に、各レビューの文ベクトルをその位置によって重み付け平均する。
これにより、文の大まかな位置関係を保持したまま
レビュー間の文ベクトルの数を固定にする。
最後に、その文書ベクトルと重み付け平均された文ベクトルを用いて
ニューラルネットワークによる分類器においてレビューを分類し
レーティング予測を行う。
%楽天トラベルのデータセットを用いた実験において、
%提案手法は従来手法\cite{fujitani15}に対して約2pp高い正答率を示した。
%レーティングの平均二乗誤差(RMSE)を元にした評価基準では、
%従来手法\cite{fujitani15}において欠点となっていたカテゴリについて
%それを上回る結果を示した。


\subsection{貢献}

実験では、提案手法と従来手法\cite{fujitani15}に加え3つの比較手法について
レーティング予測の正答率を測定し比較した。
データセットには楽天トラベルにおけるレビューデータ約33万件を用いた。
実験結果より、提案手法が従来手法\cite{fujitani15}より約2%高い正答率を示し、
提案手法が従来手法より優れていることが分かった。
提案手法と文書ベクトルのみを用いたQuocら\cite{quoc14}と同様の手法の比較により、
文書ベクトルと文ベクトルを同時に用いることが重要であることが分かった。
さらに、文ベクトルのみを用いた手法の内、文同士の位置関係を考慮したものと
考慮していないものの正答率の比較により、文ベクトルの位置関係を考慮することが
レーティング予測に重要であることが分かった。


\subsection{構成}

本論文は6節からなる。
節\fullref{sec:Introduction}は本節であり、本研究の背景、目的、
研究分野への貢献、論文の全体の構成について述べる。
節\fullref{sec:RelatedResearch}では、本研究の従来手法\cite{fujitani15}と、
提案手法がレーティング予測のために用いるいくつかの手法、研究について述べる。
節\fullref{sec:ProposedMethod}では、提案手法の基礎となる考えと
その具体的なアルゴリズムについて説明する。
節\fullref{sec:Experiments}では、提案手法と従来手法、及び、3つの比較手法を
用いた実験の実験設定と結果について説明する。
節\fullref{sec:Discussion}では、実験結果とそれによって明らかとなった
提案手法の性質について考察する。
また、明らかとなった提案手法の欠点からその改善すべき点について議論する。
節\fullref{sec:Conclusion}では、まとめと今後の予定について述べる。
