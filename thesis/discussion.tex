\section{考察} \label{sec:Discussion}

%実験結果より、提案手法と3つの比較手法全てが従来手法の正答率を上回っている。
提案手法が従来手法\cite{fujitani15}の正答率を
0.0198有意に上回っていることから、提案手法が従来手法\cite{fujitani15}より
正答率において優れていることが分かった。
また、Weighted ASVの正答率がASVの正答率を0.0029有意に上回っていることから、
文の位置関係の考慮がレーティング予測に有効であることが分かった。
さらに、提案手法がWeighted ASVに比べ有意に高い正答率を示していることから、
文書ベクトルと文ベクトルを同時に特徴量として用いることがレーティング予測に
有効であることが分かった。
これは文書ベクトルと文ベクトルがいくらか異なる特徴を学習していることを示す。

藤谷ら\cite{fujitani15}より、実験で用いたデータセットには、
食事のカテゴリにおいて0点が付与されたレビューが108,079件存在する。
また、風呂のカテゴリでは13,332件、設備のカテゴリでは2,011件存在し、
他のカテゴリでは0件である。
一般に、0点はユーザが何らかの理由でレーティング不可能と判断したことを示す。
例えば、「食事」のカテゴリならばホテルで食事を取っていない、
「風呂」のカテゴリならば別の入浴施設を利用した等である。
よって、提案手法は従来手法\cite{fujitani15}よりレビュー中の上記のような意味を
よく捉えていると考えられる。

次に、提案手法の性質について考察する。
%
表\ref{tab:AnswerRatings}より、「食事」カテゴリでレーティングが
0点である場合を除いて、正解レーティングが
0から2点であるレビューの件数は3から5点であるものに比べ
極端に少ないことが分かった。
表\ref{tab:ProposedMethodRecall}より、「食事」カテゴリでレーティングが
0点である場合と精度または再現率が計算できなかった場合を除いて、
正解レーティングが0から2点であるレビューにおける提案手法の再現率は
全て精度に比べて低い結果となった。
このため、表\ref{tab:ProposedMethodFValues}のように、
F値も0から2点のレーティングにおいて低くなってしまった。
%
特に、表\ref{tab:PredictedRatings}より、
「立地」カテゴリでレーティングが1点の場合と
「風呂」カテゴリでレーティングが0点の場合、
「設備」カテゴリでレーティングが0点の場合は全て予測レビュー件数が0件となった。
しかし、表\ref{tab:AnswerRatings}より、
これらのカテゴリとレーティングの組み合わせを持つレビューは評価データには
含まれている。
そのため、表\ref{tab:ProposedMethodRecall}のように、
そのようなカテゴリとレーティングの組み合わせでは再現率が0となってしまった。
%
以上より、提案手法はその正解レーティングを持つレビューの件数が
少ないカテゴリとレーティングの組み合わせについて、
再現率、及び、F値が低くなることが分かった。

最後に、提案手法そのものの問題点について考察する。
提案手法の問題点の一つは、レビューの素性の生成と分類のモデルが
分離していることである。
具体的には、PV-DMによってレビューの文書とそれが含む文の素性を生成する段階、
及び、それらをニューラルネットワークによって分類を行う段階の2つに分離している。
このことは、問題を2つに分けることで個々の問題を単純にしているが、
同時にいくつかの問題を伴う。
1つ目は、レビューを一つずつレーティング予測することができないことである。
提案手法では、新しいレビューを訓練データに加える場合、
全てのレビューの文書・文ベクトルを再構築する必要がある。
レビューの件数が多い場合、これは大量の計算を必要とし効率的ではない。
これは提案手法が実際に応用されるときに問題となる。
なぜなら、実際の商品レビューの件数は時間と共に増加していくためである。
2つ目は、文書や文の素性やそれらを生成するモデルのパラメータが分類時に
調整できないことである。
最大の正答率を達成するためには、これらは分類器のパラメータと同様に
分類問題に対して最適化されることが望ましい。
3つ目は、単語間の関係が分類時に十分に考慮できないことである。
単語間の関係は文書・文ベクトルによって表現されているが、
それらは分類の正答率が最大になるように表現されているとは限らない。
以上より、提案手法の素性の生成と分類のモデルは統合するべきである。
