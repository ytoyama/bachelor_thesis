\section{考察}

%実験結果より、提案手法と3つの比較手法全てが従来手法の正答率を上回っている。
提案手法が従来手法\cite{fujitani15}の正答率を
0.0198有意に上回っていることから、提案手法が従来手法\cite{fujitani15}より
正答率において優れていることが分かった。
また、Weighted ASVの正答率がASVの正答率を0.0018有意に上回っていることから、
文の位置関係の考慮がレーティング予測に有効であることが分かった。
さらに、提案手法がWeighted ASVに比べ有意に高い正答率を示していることから、
文書ベクトルと文ベクトルを同時に特徴量として用いることがレーティング予測に
有効であることが分かった。
これは文書ベクトルと文ベクトルがいくらか異なる特徴を学習していることを示す。

藤谷ら\cite{fujitani15}より、実験で用いたデータセットには、
食事のカテゴリにおいて0点が付与されたレビューが108,079件存在する。
また、風呂のカテゴリでは13,332件、設備のカテゴリでは2,011件存在し、
他のカテゴリでは0件である。
一般に、0点はユーザが何らかの理由でレーティング不可能と判断したことを示す。
例えば、食事ならばホテルで食事を取っていない、
風呂ならば別の入浴施設を利用した等である。
よって、提案手法は従来手法\cite{fujitani15}よりレビュー中の上記のような意味を
よく捉えていると考えられる。

次に、提案手法の問題点について考察する。
提案手法の問題点の一つは、レビューの素性の生成と分類のモデルが
分離していることである。
具体的には、分類すべきレビューの文書とそれが含む文の素性を生成する段階、
及び、それらをニューラルネットワークによって分類を行う段階の2つに分離している。
このことは、問題を2つに分けることで個々の問題を単純にしているが、
同時にレーティング予測をレビュー一つずつに対して行うことを不可能にしている。
また、文書や文の素性を事前に生成するモデルのパラメータは、
実際には最大の正答率を達成するため
分類器のパラメータと同様に分類問題に対して最適化されることが望ましい。

%今後は、これらの問題を解決するために、文書や文の分散表現を生成する過程を
%ニューラルネットワークによる分類器に統合する。
%これによって、学習方法の柔軟性を高めると共にさらなる正答率の向上を目指す。
