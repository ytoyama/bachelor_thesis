\section{考察} \label{sec:Discussion}

まず、提案手法と従来手法を正答率及び正解レーティングと予測レーティングの
RMSEについて比較する。
表\ref{tab:AccuraciesOfMethods}より、提案手法が\rival の
正答率を0.0198有意に上回っている。
よって、提案手法が従来手法より正答率において優れていることが分かった。
さらに、図\ref{tab:AccuraciesPerCategory}においても、
提案手法の正答率は従来手法を全てのカテゴリにおいて上回っている。
レーティングのRMSEについては、図\ref{fig:RMSEsPerCategory}より、
提案手法のRMSEが従来手法のそれを全てのカテゴリにおいて下回っている。
よって、レーティングのRMSEにおいても提案手法が従来手法より優れていることが
分かった。
ここで、実験に用いたレビューデータの特徴から提案手法と従来手法の違いについて
考察する。
藤谷ら\cite{fujitani15}より、実験で用いたデータセットには、
「食事」のカテゴリで0点が付与されたレビューが108,079件存在する。
さらに、0点が付与されたレビューは「風呂」のカテゴリでは13,332件、
設備のカテゴリでは2,011件、他のカテゴリでは0件存在する。
一般に、0点はユーザが何らかの理由でレーティング不可能と判断したことを示す。
例えば、「食事」のカテゴリならばホテルで食事を取っていない、
「風呂」のカテゴリならば別の入浴施設を利用した等である。
よって、提案手法は従来手法よりレビュー中の上記のような意味を
よく捉えていると考えられる。
また、表\ref{tab:AnswerRatings}と表\ref{tab:PredictedRatings}を比較すると、
正解レーティングが0点であるレビューが0件であるカテゴリにおいて、
提案手法は0点を1件も予測していない。
このことからも、0点が1から5点のレーティングと異なり
レーティングの度合いを示すものではないということを、
提案手法は上手く考慮できているといえる。

次に、文の位置関係の考慮がレーティング予測に正答率の向上に
有効であるかを検証する。
表\ref{tab:AccuraciesOfMethods}より、Weighted ASVの正答率がASVの正答率を
0.0029有意に上回っている。
ここで、ASVはレビュー内の文ベクトルを平均しただけの素性を
分類器の入力としている。
すなわち、文の特徴は考慮しているがその位置関係は考慮していない。
また、Weighted ASVは文ベクトルを位置によって重み付け平均したベクトルを
素性としている。
すなわち、文の特徴とその大まかな位置関係を考慮している。
以上より、文の位置関係を考慮することはレーティング予測に有効であることが
分かった。

次に、文書ベクトルと文ベクトルを同時に素性として用いることが
正答率の向上に有効であるかを検証する。
表\ref{tab:AccuraciesOfMethods}より、提案手法がDVやWeighted ASVに比べ
有意に高い正答率を示している。
提案手法とDVの手法における差は重み付け平均された文ベクトルを素性として
用いるか用いないかである。
提案手法とWeighted ASVの差は文書ベクトルを素性として
用いるか用いないかである。
以上より、文書ベクトルと文ベクトルを同時に特徴量として用いることが
レーティング予測に有効であることが分かった。
また、このことは文書ベクトルと文ベクトルがいくらか異なる特徴を
学習していることを示す。
なぜならば、文書ベクトルと文ベクトルが同じ特徴を学習していた場合、
提案手法とDVの正答率における有意差は無くなるはずだからである。
ただし、このとき提案手法とWeighted ASVの正答率における有意差は
一般に無くならない。
なぜならば、文ベクトルを分類器において評価する方法として重み付け平均が
最適とは限らないからである。

次に、提案手法のその他の性質について考察する。
%
表\ref{tab:AnswerRatings}より、「食事」カテゴリでレーティングが
0点である場合を除いて、正解レーティングが
0から2点であるレビューの件数は3から5点であるものに比べ少ないことが分かった。
この正解レーティングが0から2点であるレビューについて、
表\ref{tab:ProposedMethodRecall}より、「食事」カテゴリでレーティングが
0点である場合と精度または再現率が計算できなかった場合を除いて、
提案手法の再現率は全て精度に比べて低い結果となった。
このため、表\ref{tab:ProposedMethodFValue}のように、
F値も0から2点のレーティングにおいて他のレーティング値に比べ
低くなってしまった。
%
特に、表\ref{tab:PredictedRatings}より、
「立地」カテゴリでレーティングが1点の場合と
「風呂」カテゴリでレーティングが0点の場合、
「設備」カテゴリでレーティングが0点の場合は全て予測レビュー件数が0件となった。
しかし、表\ref{tab:AnswerRatings}より、
これらのカテゴリとレーティングの組み合わせを持つレビューは評価データに
含まれている。
そのため、表\ref{tab:ProposedMethodRecall}のように、
そのようなカテゴリとレーティングの組み合わせでは再現率が0となってしまった。
%
以上のように、提案手法は0から2点のレーティングにおいて、
再現率、及び、F値が低くなってしまうことが分かった。

最後に、提案手法そのものの問題点について考察する。
提案手法の問題点の一つは、レビューの素性の生成と分類のモデルが
分離していることである。
具体的には、パラグラフベクトルによってレビューの文書とそれが含む文の素性を
生成する段階、及び、それらをニューラルネットワークによって分類を行う段階の
2つに分離している。
このことは、問題を2つに分けることで個々の問題を単純にしているが、
同時にいくつかの問題を伴う。
1つは、レビューを一つずつレーティング予測することができないことである。
提案手法では、新しいレビューを訓練データに加える場合、
評価データを含め全てのレビューの文書・文ベクトルを再構築する必要がある。
レビューの件数が多い場合、これは大量の計算を必要とし効率的でない。
これは提案手法が実際に応用されるときに問題となる。
なぜなら、実際の商品レビューの件数は時間と共に増加していくためである。
2つ目は、文書や文の素性、及び、それらを生成するモデルのパラメータが分類時に
調整できないことである。
最大の正答率を達成するためには、これらは分類器のパラメータと同様に
対象とする分類問題に対して最適化されることが望ましい。
3つ目は、単語等のより小さな言語要素間の関係が
分類時に十分に考慮できないことである。
単語間の関係は文書・文ベクトルによってある程度表現されている。
しかし、それらは分類の正答率が最大になるように表現されているとは限らない。
以上の問題点から、提案手法の素性の生成と分類のモデルは
統合するべきであると考えられる。
なぜなら、モデルを統合すれば1つ目と2つ目の問題点が解消できるためである。
3つ目の問題点も、レビューの文書中における単語や文字等の
小さな言語要素の特徴からレーティングを予測することで対処することができる。
統合されたモデルには、入力間出力間の複雑な関係を考慮できる
ニューラルネットワークのようなモデルを引き続き用いることが適切であると
考えられる。
