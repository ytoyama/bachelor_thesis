\section{結論} \label{sec:Conclusion}

\subsection{まとめ}

本研究では、多カテゴリにおけるレーティング予測について、
レビュー全体の文書ベクトルに加え重み付け平均された文ベクトルを用いた手法を
提案した。

楽天トラベルのレビューデータを用いた実験では、
提案手法が従来手法\cite{fujitani15}より高い正答率を示した。
提案手法及び3つの比較手法の比較により、
レビュー内の文の並びがレーティング予測に重要であること、及び、
文書ベクトルと文ベクトルがレビューのいくらか異なる特徴を捉えていることが
分かった。
また、使用したデータセットに対する提案手法の性質として、
評価データの内レビュー件数の少ない0点から2点のレーティングにおいて
再現率、及び、F値が低くなってしまうことが分かった。


\subsection{今後の予定}

今後は、単語間や文字間等の言語要素間の
より多様で複雑な関係を考慮することを課題とする。
考察より、このためには各レビューの素性を生成するモデルと
分類を行うモデルを1つに統合する必要がある。
なぜならば、モデルが分かれていることによって
小さな言語要素同士の関係を予測時に考慮できないためである。
統合されたモデルには、単語間や文字間の複雑な関係を考慮するため
ニューラルネットワークを用いる。
また、これによって、訓練レビューを一つずつ学習できるようになり、
提案手法のように新しい訓練レビューが追加される毎に訓練レビュー全てについて
素性を再計算する必要が無くなる。
さらに、単語・文字等の小さな言語要素の特徴から文書・文等のより大きな言語要素の
特徴をニューラルネットワークによって自動的に構成すれば、
文書・文間の関係も同時に考慮できると考えられる。
