\section{結論}

本研究では、多カテゴリにおける評判分類問題について、
レビュー全体の文書ベクトルに加え重み付け平均された文ベクトルを用いた手法を
提案した。

実験では、提案手法が従来手法\cite{fujitani15}より高い正答率を示した。
また、比較手法の結果より、
レビュー内の文の並びが評判分類に重要であること、及び、
文書ベクトルと文ベクトルがレビューのいくらか異なる特徴を捉えていることが
分かった。

今後の課題は言語要素間のより多様で複雑な関係を考慮することである。
このためには、各レビューの意味表現を生成するモデルと
分類を行うモデルを1つに統合する必要がある。
なぜならば、モデルが分かれていることによって
単語や文字などのより小さな言語要素同士の関係を分類時に考慮できないためである。
モデルの統合によって、学習手法の柔軟性を高めると共に
さらなる正答率の向上を目指す。


%今後の課題は、提案手法の中で2つに分かれているモデルの統合である。
%
%提案手法は、分類すべき文書とそれが含む文の分散表現を生成する段階、及び、
%それらを用いて分類を行う段階の2つの段階に分かれている。
%このことは、問題を2つに分けることで個々の問題を単純にしているが、
%同時に文書の分類を一つずつ行うことを難しくしている。
%また、文書や文の分散表現を事前に生成するための
%PV-DMのパラメータは、実際には最大の正答率を達成するため
%分類器のパラメータとして最適化されることが望ましい。
%
%今後は、これらの問題を解決するために、文書や文の分散表現を生成する過程を
%ニューラルネットワークによる分類器に統合する。
%これによって、学習方法の柔軟性を高めると共にさらなる正答率の向上を目指す。
