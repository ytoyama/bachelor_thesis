\section{結論} \label{sec:Conclusion}

\subsection{まとめ}

本研究では、多カテゴリにおけるレーティング予測について、
レビュー全体の文書ベクトルに加え重み付け平均された文ベクトルを用いた手法を
提案した。

楽天トラベルのレビューデータを用いた実験では、
提案手法が従来手法\cite{fujitani15}より高い正答率を示した。
さらに、カテゴリ別の正答率、カテゴリ別のレーティングのRMSEを用いた
評価基準においても、提案手法は従来手法より優れていることが分かった。
文ベクトルのみを用いた2つの比較手法の実験結果より、
レビューの文書内の文の並びがレーティング予測に重要であることが分かった。
提案手法及び2つの比較手法の実験結果より、
文書ベクトルと文ベクトルを同時に用いることが正答率の向上に有効であり、また、
文書ベクトルと文ベクトルがレビューのいくらか異なる特徴を捉えていることが
分かった。
また、実験において使用したデータセットに対する提案手法の性質として、
0点から2点のレーティングにおいて再現率、及び、F値が低くなってしまうことが
分かった。


\subsection{今後の予定}

今後は、単語や文字等のより小さな言語要素間の複雑な関係を考慮することを
課題とする。
考察より、このためには各レビューの素性を生成するモデルと
分類を行うモデルを1つに統合する必要がある。
なぜならば、モデルが分かれていることによって
小さな言語要素同士の関係を予測時に考慮できないためである。
統合されたモデルには、単語や文字間の複雑な関係を考慮するため\nn を用いる。
また、これによって、訓練レビューを一つずつ学習できるようになり、
提案手法のように新しい訓練レビューが追加される毎にレビューデータ全てについて
素性を再計算する必要が無くなる。
さらに、単語・文字等の小さな言語要素の特徴から
文書・文等のより大きな言語要素の特徴を
ニューラルネットワークによって自動的に構成すれば、
文書・文間の関係も同時に考慮できると考えられる。
修士研究では、これらを達成することによってレーティング予測の正答率を
さらに向上させることを目指す。
